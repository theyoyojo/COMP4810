\documentclass{article}
\usepackage{graphicx, url}
\usepackage[margin=1in]{geometry}

\author{Joel Savitz}

\begin{document}


\begin{titlepage}
   \begin{center}
       \vspace*{1cm}

	\textbf{Fedora Linux Development for the Raspberry Pi Platform \\
		\small Backwards compatibility for a popular GPIO API}

       \vspace{0.5cm}
        A Short Report and Specification
            
       \vspace{1.5cm}

       By \\
       \textbf{Joel Savitz}

       \vfill
            
       Submitted in partial fulfillment
       of the requirements of the \\
       Commonwealth Honors College \\
       University of Massachusetts Lowell \\
       2020
            
       \vspace{0.8cm}
     
            
       Honors Mentor: Professor Bill Moloney, Department of Computer Science

       Committee Member: Jeff Brown, Red Hat

% \noindent\begin{tabular}{ll}
% 	\\[4ex]
% 	\makebox[4in]{\hrulefill} & \makebox[1.5in]{\hrulefill} \\
% 	Joel Savitz, Author & Date \\[5ex]
% 	\makebox[4in]{\hrulefill} & \makebox[1.5in]{\hrulefill} \\
% 	Professor Bill Moloney, Honors Mentor & Date \\[5ex]
% 	\makebox[4in]{\hrulefill} & \makebox[1.5in]{\hrulefill} \\
% 	Jeff Brown, Committee Member & Date \\[5ex]
% \end{tabular}

            
   \end{center}
\end{titlepage}

\begin{abstract}

This document summarizes
the primary achievement
of my honors project.
First, I give some context
for the origin of the project
and reasoning for its direction.
Then, I describe the status of the project,
and include the functional and technical specification
of the primary software output.

\end{abstract}

\section{Background}

In the early spring of 2019,
I walked into a meeting.
As a co-op intern at Red Hat,
I was new to full time work
in the software engineering industry
and attending random optional meetings
was an interesting proposition to me.
As it turns out,
I walked into the founding meeting
of an initiative to expand
the Red Hat Metro Boston Research Interest Group
to the University of Massachusetts Lowell.
They had projects and mentors
and were in need of students.
I had friends in need
of projects and mentors
Their problem was my solution,
and my problem was their solution.
After a few somewhat vague proposals
were politely rejected,
a few friends and I were introduced to Jeff Brown.
He teaches an IoT class at UML during the spring semester,
and he wanted to improve Fedora usability
on the Raspberry Pi platform,
since as a Red Hat associate,
it was far more appropriate
than the existing alternatives,
despite a severe and overwhelming
lack of community support and development.

Soon after the initial meeting,
I vanished off the face of the earth
due to my contraction of Mononucleosis.
Sporadic meetings, emails, and conversations with friends
led to a more concrete idea of the opportunity
in the late summer of 2019.
Professor Moloney approved,
and the initiative was established.

Jeff and I had a short meeting
with Peter Robinson,
the maintainer of
Fedora for ARM architecture,
and he recommended several areas
of potential development.

One of these was in the area
of GPIO library compatibility.
As a summary the problem,
and our solution
consider the following segment
of the project README document:

\begin{quote}
RPi.GPIO requires non-standard kernel patches that expose the GPIO registers to userspace via a character device /dev/gpiomem. As this is not supported by the mainline Linux kernel, any distribution targeting Raspberry Pi devices running the mainline kernel will not be compatible with the RPi.GPIO library. As a large number of tutorials, especially those targeted at beginners, demonstrate use of the RPi's GPIO pins by including RPi.GPIO syntax, this incompatibility limits users to distributions build on a special downstream kernel maintained by the Rapberry Pi foundation. We would like to enable beginners on any Linux distribution by allowing them to follow easily available tutorials.

Using the provided module, one will be able to write python code to use the Raspberry Pi's GPIO pins as if they were using the API implemented by RPi.GPIO, but instead using libgpiod's python bindings. libgpiod provides a straightforward interface for interacting with GPIO pins on supported devices via the mainline Linux kernel interface \cite{underground}.
\end{quote}

The implementation of this library
was my primary achievement for this honors project.
I owe a great deal of credit
to Fabrizio D'Angelo
for the design and implementation
of this library
as much of the initial work
was done in the style of pair programming
during late Wednesday night sessions
in Dandeneau Hall on UML North campus.

The initial implementation of this project
took place last fall,
but I developed it to the point
of feature-equivalence
with RPi.GPIO in early summer 2020.
At this time,
I decided to write
a functional and technical specification
to make the open source project more accessible
to potential contributors
and to more effectively and explicitly
design the software.
The document specifies the API behavior of the library
by documenting the existing behavior of RPi.GPIO.
I define correct behavior of the library
to mean replication of the behavior of RPi.GPIO 0.7.0
to allow the library to be used as a drop-in replacement.

In recent discussions with fellow project participants,
we have discussed extending the library
to support additional features
long planned by the author of the original RPi.GPIO library,
yet indefinitely delayed.
If we move forward with this,
we will extend the specification as well.

The library is available for usage
in the PyPi python package repository
under the name ``RPi.GPIO2''.
The Linux command to accomplish this task
is \texttt{pip install RPi.GPIO2}.
If installed in this manner,
one must ensure that they also
install the libgpiod python bindings,
as the python package manager
does not do this automatically.
I am currently in the process
of packaging the library for
the official Fedora package repositories,
and that version will automatically
install all necessary dependencies.

The latest version of the specification follows in its entirety.
A newer version, if available, is accessible at
\url{https://github.com/underground-software/python3-libgpiod-rpi/tree/master/spec/spec.pdf}.

\input{spec}

\bibliographystyle{acm}

\bibliography{report}

\end{document}
